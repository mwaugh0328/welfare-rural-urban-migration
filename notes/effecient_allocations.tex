\documentclass[12pt,pdftex]{article}
\usepackage[pdftex]{graphicx,color}
\usepackage{setspace,palatino,multirow}
\usepackage{amsmath,amssymb}
\usepackage{titlesec}
\usepackage{lscape}
%\usepackage{subfigure}
\usepackage{threeparttable}
\usepackage{natbib}
\bibliographystyle{ecta}
\usepackage{cite}
\usepackage{booktabs}
\usepackage{subcaption}
\usepackage{pdflscape}
\usepackage{afterpage}
\usepackage{xcolor}
\usepackage{rotating}

\definecolor{nblue}{RGB}{0,0,128}

\usepackage[pdftex,colorlinks=true, bookmarks=false,
pdfstartview={XYZ null null 0.65},
pdftitle={},
pdfauthor={ Michael E. Waugh},
pdfkeywords={},
colorlinks=true,linkcolor=darkgray,citecolor=darkgray,urlcolor=darkgray,
breaklinks]{hyperref}

\newcounter{saveeqni}%
\newcounter{saveeqn01i}%
\newcommand{\alpheqni}{\setcounter{saveeqni}{\value{section}}%
%\setcounter{saveeqn01i}{\value{subsectioni}}%
\renewcommand{\theequation}
    {\alph{saveeqni}\mbox{.\arabic{equation}}}}%
\newcommand{\reseteqni}{\setcounter{equation}{\value{saveeqni}}%
\renewcommand{\theequation}{\arabic{equation}}}%

\newtheorem{as}{Assumption}
\newtheorem{reg}{Regularity Condition}
\newtheorem{conjecture}{Conjecture}
\newtheorem{corr}{Corollary}
\newtheorem{df}{Definition}
\newtheorem{lemma}{Lemma}
\newtheorem{prp}{Proposition}
\newtheorem{rmk}{Remark}
\newenvironment{prf}{{\bf Proof}}{\hfill { }}
\newtheorem{proposition}{Proposition}

\DeclareMathOperator*{\plim}{plim}
\DeclareMathOperator*{\umax}{max}

\special{papersize=8.5in,11in}
\onehalfspacing
\setlength{\parindent}{0.1em}
\setlength{\parskip}{.09in}
\textwidth15.75cm
\evensidemargin 1.5in
\oddsidemargin 1.5in
\topmargin 8.5cm
\textheight 10in
\hyphenation{over-lapping}

\titleformat{\section}{\color{black}\large\bf}{\color{black}{\thesection.}}{.25cm}{}
\titleformat{\subsection}{\color{black}\normalsize\bf}{\thesubsection.}{.5em}{}
\titleformat{\subsubsection}{\color{black}\normalsize\bf}{\thesubsubsection.}{.5em}{}

\titlespacing{\section}{0pt}{*1.5}{*.5}
\titlespacing{\subsection}{0pt}{*1.5}{*.5}
\titlespacing{\subsubsection}{0pt}{*1.5}{*.5}

\def\thesection{\arabic{section}}
\def\thesubsection{\arabic{section}.\arabic{subsection}}
\def\thesubsubsection{\arabic{section}.\arabic{subsection}.\Alph{subsubsection}}

\def\citeapos#1{\citeauthor{#1}'s (\citeyear{#1})}

\renewcommand{\arraystretch}{1.1}
\usepackage[margin=2cm]{geometry}

\begin{document}
\begin{onehalfspacing}
\newpage

\normalsize



\section{Rural Urban Migration and Efficient Allocation}\label{sed:intro}

This section solves and computes the efficient allocation in our model economy. Up to this point, we have measured the welfare gains from a particular policy intervention|a subsidy to induce more seasonal migration. In all the instances we studied (one-time partial equilibrium, permanent general equilibrium with distortionary taxes) the welfare gains from these interventions result from the provision of an opportunity to that allows households to better smooth their consumption by moving across space.

However, all these experiments leave open the normative question about and how much households \emph{should} migrate when markets are complete and the economy is undistorted. In other words, what is the first-best allocation and how does it compare to the aforementioned policy interventions? This is what we study below.

\subsection{The Social Welfare Function}

To study these normative questions, we must take a stand on the social welfare function. We focus on a utilitarian planner that places equal weight on households. Define the social welfare function as
\begin{align}
\mathcal{W^{SP}} = \sum_{t=0}^{\infty}\sum_{j} \int\limits_{z} \int\limits_{s} \int\limits_{x} \int\limits_{\nu} \beta^{t} \ u_{j}(c_{j}(z, s, x, t), x, \nu) \lambda_{j}(z, s, x, \nu, t) dz \ ds \ dx \ d\nu.
\label{eq:sp-social_welfare}
\end{align}
Here social welfare is the average value of households utility across locations $j$, productivity states $z$ and $s$, experience $x$, and preference shocks $\nu$. The average is computed with respect to the measure of households $\lambda_{j}(z, s, x, \nu, t)$ with those shock states, experience levels, and preference shocks at date all dates $t$. Utility depends directly upon the consumption allocation $c_{j}(z, s, x, t)$ assigned to the household given their states, date, and location; utility also depends directly on the location $j$, experience level via the $\bar u$, and the idiosyncratic preference shock.

%At a high-level, in the planning problem the idiosyncratic preference shock acts like a randomization device or lottery that allows the planner to assign (for a given set of states) some fraction of households to migrate and some to stay. The difference relative to a pure lottery is that these preference shocks are valued by households, so the assignment is such that only those households with the highest relative valuation to migrate, for example, actually migrate.  Appendix A details this argument and it's application with Type 1 extreme value shocks.

We cast the Planners Problem in terms of the planner choosing consumption allocations and migration probabilities. The formulation of the problem in terms of migration probabilities is novel and, hence, we discuss this part a bit more. To cast the problem in terms of migration probabilities, we integrate out the preference shocks conditional on a set of migration probabilities for each household state. Given a set of states, these migration probabilities prescribe an assignment of those households with the largest relative preference shock to migrate or not. And given this prescription, we compute expected utility across the preference shock. So given set of states $j, z, s, x, t$, utility is
\begin{align}
u(c_{j}(z,s, x, t), x) + E[ \ \nu \ | \ \big\{\mu_{j',j}(z,s,x,t)\big\}_{j'} ].
\label{eq:utility-shocks}
\end{align}
where $\mu_{j',j}(s,x,t)$ is the migration probability of households in location $j$ to location $j'$ and then $E[ \ \nu \ | \ \big\{\mu_{j',j}(z,s,x,t)\big\}_{j'} ]$ is the expected value of the presence shock conditional on the migration probabilities. To understand this a bit better, for example, if all people migrate location $j$ to location $j'$, then this value is the unconditional mean of a Type 1 extreme value shock. Now we can write the social welfare function as
\begin{align}
\mathcal{W^{SP}} = \sum_{t=0}^{\infty}\sum_{j} \int\limits_{z} \int\limits_{s} \int\limits_{x} \beta^{t} \bigg \{ u(c_{j}(s, x, t), x) + E[ \ \nu \ | \ \big\{\mu_{j',j}(s,x,t)\big\}_{j'}] \bigg \} \lambda_{j}(s, x, t) dz \ ds \ dx.
\label{eq:sp-social_welfare2}
\end{align}
So period utility equals the consumption component of utility plus the expected value of the preference shock which depends upon the chosen migration probabilities.

\subsection{The Law of Motion and Feasibility}

The Planning Problem maximizes (\ref{eq:sp-social_welfare2}) subject to the law of motion describing how the population evolves across states and locations and then how many resources the are available, i.e., feasibility. We describe each of these aspects of the environment below.

\textbf{Law of Motion.} The law of motion describing how the measure of households evolves across states and locations is
\begin{align}
\lambda_{j}(z, s', x', t+1)  & =  \int\limits_{s} \int\limits_{x}  \mu_{j,j}(z, s,x,t)\pi(s',s) \varphi(x',x, j) \lambda_{j}(z, s, x, t)  \ ds \ dx  \  \label{eq:planner_law_motion} \\
& +  \sum_{j' \neq j} \int\limits_{s} \int\limits_{x} \mu_{j,j'}(z, s,x,t) \pi(s',s) \varphi(x',x, j') \lambda_{j'}(z, s, x, t)  \ ds  \ dx. \nonumber
\end{align}
where $\pi(s',s)$ describes the transition across transitory states and $\varphi(x',x, j')$ describes the transition in experience. This equation says, given the current distribution  $\lambda_{j'}(z, s, x, t)$ in location $j'$, the measure of households in location $j$ with productivity shock state $s'$ and experience level $x'$ next period reflects the migration probabilities of households in each location (the $\mu$'s), how their productivity evolves over time ($\pi's$), and how their experience $\varphi(x',x, j)$ evolves. 

\textbf{Labor Supply, Aggregate Production, and the Resource Constraint.} Given a distribution of households, the effective labor units in the urban and rural area are
\begin{align}
N_{u,t} =& \sum_{j = [\mbox{\tiny urban}, \mbox{\tiny seas}]}\int\limits_{z} \int\limits_{s} \int\limits_{x} \  z s^{\gamma} \ \lambda_j(z, s, x, t) \ dz \ ds \ dx, \nonumber 
\\
\nonumber \\
N_{r,t} =& \int\limits_{z} \int\limits_{s} \int\limits_{x} \ s \ \lambda_{\mbox{\tiny rural}}(z, s, x, t)\ dz \ ds \ dx \ .
\label{eq:planner_labor_supply}
\end{align}
with the urban area includes the seasonal and permanent urban workforce. Aggregate production of the final good is
\begin{align}
Y_t = A_u N_{u,t} + A_{r,t} \left(N_{r,t} \right)^{\phi}.
\label{eq:planner_value_production}
\end{align}
Combining the amount of resources available in (\ref{eq:planner_value_production}) with the consumption and moving decisions we have the following resource constraint:
\begin{align}
Y_t\  \geq \ \sum_{j} \int\limits_{z} \int\limits_{s} \int\limits_{x} c_{j}(z, s, x, t) \lambda_{j}(z, s, x, t) \ ds \ dx + \sum_{j}\sum_{j'} \int\limits_{z} \int\limits_{s} \int\limits_{x}  m_{j',j} \ \mu_{j',j}(z,s, x, t) \lambda_{j}(z, s, x, t) \ ds \ dx.
\label{eq:planner_income_side_gdp}
\end{align}
The left hand side is production and follows from the production function, the distribution of households (\ref{eq:planner_law_motion}) and how it evolves, and  labor supply (\ref{eq:planner_labor_supply}). Production must be greater than or equal to consumption which is the first term on the righthand side of (\ref{eq:planner_income_side_gdp}) and the moving costs associated with the migration of households across locations which is the second term on the righthand side. Here we compactly sum across all $j'$ and $j$ location pairs with the notion that the moving cost for staying in a location is zero, i.e., $m_{j,j} = 0$.

\subsection{The Planner's Problem and the Efficient Allocation}

The Planner's Problem is to chose consumption allocations and migration probabilities to maximize the social welfare function in (\ref{eq:sp-social_welfare2}) subject to the constraints imposed by feasibility (\ref{eq:planner_income_side_gdp}), the law of motion (\ref{eq:planner_law_motion}) and that the migration probabilities sum to one. In math, the problem is
\begin{align}
\small
& \max \ \sum_{t=0}^{\infty}\sum_{j} \int\limits_{z} \int\limits_{s} \int\limits_{x} \beta^{t} \bigg \{ u(c_{j}(z, s, x, t), x) + E[\ \nu \ | \ \mu_{j',j}(z, s,x,t)] \bigg \} \lambda_{j}(z, s, x, t) \ dz \ ds \ dx \label{eq:planner_problem} \\
& \mbox{subject to} \nonumber\\[.75em]
& \ \ \ \ \mbox{feasibility}, \ \ (\ref{eq:planner_income_side_gdp}), \ \ \ \ [\ \mathbf{\chi(t)} \ ], \nonumber \\[.75em]
& \ \ \ \ \mbox{law of motion}, \ \ (\ref{eq:planner_law_motion}), \ \ \ \ [\ \mathbf{\chi_{3j}(s, x, t)} \ ], \nonumber \\[.75em]
& \ \ \ \ \mbox{migration probabilities}, \ \ \sum_{j'} \mu_{j',j}(s,x,t) = 1,  \ \ \ \ [\ \mathbf{\chi_{2j}(s, x, t} \ ], \nonumber \\[.75em]
& \ \ \ \ \mbox{and an inital condition}, \ \ \ \ \lambda_j(s, x,0). \nonumber
\end{align}
where the $\chi$ terms in brackets denote the Lagrange Multipliers associated with constrained optimization problem. We the call the allocation induced by the solution to \ref{eq:planner_problem} as the \emph{efficient allocation}. After deriving the necessary conditions associated with \ref{eq:planner_problem}, Proposition \ref{prp:efficient} characterizes the consumption allocations and migration probabilities associated with these necessary conditions.

\begin{proposition}[\textbf{Efficient Consumption and Migration}] \label{prp:efficient} The consumption and migration rates that solving the Planners Problem in (\ref{eq:planner_problem}) are: Consumption allocations equate the marginal utility of consumption in all locations, productivity and experience states for each date $t$:
{\small
\begin{align}
u'(t) = u'(c_{j}(z, s, x, t)) = u'(c_{j'}(z, s', x', t)) \ \forall \ j, \ z, \ s, \ x, \ \mbox{and}  \ j', \ s', \ x'.
\label{eq:foc_planner2}
\end{align}
}
Migration probabilities satisfy
{\footnotesize
\begin{align}
& \mu_{j'j}(z,s,x,t)  =  \exp \left(\frac{- u'(t) \ m_{j',j} + \beta \ \mathbb{E}_{s,x}\left[\chi_{3j'}(z, t+1)\right]}{\sigma} \right)  \Bigg / \sum_{j'} \exp \left( \frac{- u'(t)\ m_{j',j} + \beta \  \mathbb{E}_{s,x}\left[\chi_{3j'}(z, t+1) \right]}{\sigma} \right), \label{eq:migration_prob}
\end{align}
}
with the multiplier $\chi_{3j'}(s, x, t+1)$ satisfying the following recursive relationship
{\small
\begin{align}
& \ \ \chi_{3j'}(z, s, x, t+1) =  u_{j'}(x, t+1) +  u'(t+1) \kappa_{j'}(z, s,x,t+1) + \beta \mathbb{E}\left[\chi_{3}(z, t+2) \right], \label{eq:dynamic_multiplier}
\end{align}}
where
{\small
\begin{align}
& \ \ \kappa_{j'}(z, s',x',t+1) = \mbox{\texttt{mpl}}_{j'}(z,s',t+1) - c(z, s',x',t+1) - \sum_{j''}  m_{j'',j'} \ \mu_{j'',j'}(z, s', x', t+1). \label{eq:kappa}
\end{align}}
\end{proposition}
Proposition \ref{prp:efficient} has a lot of content to unpack. First, consumption allocations equalize the marginal utility of consumption across all locations, productivity and experience states for each date $t$. This is the standard ``full risk sharing'' result in complete markets allocations that one would see in, e.g., the textbook of \citet{ljsa03}.

There is, however, an important point of departure within the context of our model and specifically the spatial dimension. Equation (\ref{eq:foc_planner2}) does not imply that the planner equates the level of consumption across households. The issue is that in our model, a household's experience and the disutility of being in the urban area affects the households' marginal utility of consumption. Mechanically this necessitates that the planner compensates non-experienced rural-urban migrants with higher consumption to equate the marginal utility of consumption across \emph{all} households.

Two observations follow from the fact that the level of consumption may be different across locations in the efficient allocation. First, when moving households across space, the planner must factor in the differential in the consumption cost associated with moving a non-experienced, rural household to the urban area, not just the moving cost. Second, and more broadly, it is a clear example where gaps in consumption (either in the cross-section or when households move as in the BCM experiment) arise in an efficient allocation because of differences in preferences. In other words, consumption gaps do not necessarily imply the allocation is inefficient or that there is scope for gains to reallocate households across space.

While the migration probabilities (\ref{eq:migration_prob}) take on a familiar $\exp$ sum form that is facilitated by the Type 1 extreme value shocks, how the migration probabilities connect with fundamentals in (\ref{eq:dynamic_multiplier}) and (\ref{eq:kappa}) are novel. 

The migration probabilities reflect two components which reflect societies valuation of the cost and benefits of a move. First, is the cost and it's evaluated at the (common) marginal utility of consumption $-u'(t) \ m_{j',j}$. So if the migration cost from $j$ to $j'$ is high, less people move. If aggregate resources are scarce, then less people move. 

The benefit of migration is a forward-looking, dynamic component which reflects societies valuation of a household with shock states $z,s$ and experience in location $j'$. Let's upack this a bit more. The second component is the expected value of the multiplier for a given destination $\chi_{3j'}(z, t+1)$ with the expectation taken across the shock states and experience. This multiplier (\ref{eq:dynamic_multiplier}) takes on a recursive formulation where the multiplier today equals utility, the $\kappa$ term (more on this below), and then the expected, discounted multiplier tomorrow taking into account all the different possible moving options out of location $j'$. What this is says is if the social value of an household is, in expectation, high in location $j'$ the Social Planner wants more households to move there.

What determines the social value of a location? It's about what a household with those states receives from the Planner vs. what is produced in return. What it receives is the first term in (\ref{eq:dynamic_multiplier}): a households level of utility evaluated at the optimal (full insurance) consumption allocation and the expected preference shock. Then the $\kappa$ term in (\ref{eq:kappa}) reflects the net resources produced by the household, i.e. it's marginal product of labor net of consumption received and net expected moving costs|all evaluated at the marginal utility of consumption. 

All of this gives rise to intuitive features of the allocation. If a household is expected to be have a high marginal product of labor in location $j$ it moves them. If a household requires relatively more consumption in that location and properly equate the marginal utility of consumption (which would arise when the household is inexperienced) the Planner is likely to move that the household. The planner also thinks through the moving cost associated with that location as well through this term $\sum_{j'}  m_{j',j} \ \mu_{j',j}(z, s, x, t+1)$ which is a migration weighted average of the moving costs. So it indexes how hard (or not) it is to move a household out of that location in the future. So if a location is ``expensive'' move out, less people move in.

How does this allocation relate to the competitive equilibrium? This is less obvious, but some general thoughts can be arrived at. A key theme is that in the Planning Problem there is a common, societal valuation on the costs and benefits of moving and these valuations induce the efficient allocation. In the competitive equilibrium, incomplete markets gives rise to differences in the marginal utility of consumption and, hence, different valuations on the costs and benefits of a move. This could give rise to situations in which there is either too much migration or too little. And policy interventions such as migration subsidies could push the allocation towards or away from the socially optimal outcome. We explore these issues next.


\subsection{The Efficient Allocation}

This section computes the efficient allocation given the calibrated parameters from section. Like in our competitive equilibrium, we focus on stationary equilibria which give rise to time invariant consumption allocations and migration probabilities. Appendix ??? describes the computational details and various robustness checks.\footnote{At a high-level, we first guess consumption allocations and marginal product of labor per efficiency unit. Given this guess, we recursively compute the multipliers $\chi_{3j'}$ and the migration probabilities via fixed point iteration. Then we construct aggregates and check if the initial guesses were consistent with feasibility. If not, update the guess until feasibility is satisfied. This approach worked well.}

Table XX reports the results and comparisons to various benchmarks. 


\newpage

\bibliography{./migration_refs}

\newpage

\appendix

\section{Cutoffs, Migration Probabilities, and Expected Utility}\label{appendix:migration_probs}




\end{onehalfspacing}
\end{document} 